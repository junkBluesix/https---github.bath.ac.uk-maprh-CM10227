\documentclass[12pt, oneside]{article}   	% use "amsart" instead of "article" for AMSLaTeX format
\usepackage{geometry}                		% See geometry.pdf to learn the layout options. There are lots.
\geometry{a4paper}                   	% ... or a4paper or a5paper or ... 
%\geometry{landscape}                	% Activate for rotated page geometry
%\usepackage[parfill]{parskip}    	% Activate to begin paragraphs with an empty line rather than an indent
\usepackage{graphicx}		% Use pdf, png, jpg, or eps§ with pdflatex; use eps in DVI mode
					% TeX will automatically convert eps --> pdf in pdflatex		
\usepackage{amssymb}
\usepackage{listings}

\title{CM10277: Principles of Programming I \\ Coursework Part 1: Java SRPN}
\author{Dr. Rachid Hourizi \& Dr. Michael Wright \\ Dept. of Computer Science \\ University of Bath}
\date{}							% Activate to display a given date or no date

\begin{document}
\maketitle
\section{Introduction}
The coursework of this unit consists of three parts: the lab sheets and two larger courseworks. This document provides the specification for the second larger coursework: \textbf{Codewalk}.

Questions regarding the coursework can always be posted on the Moodle Forum and programming1@lists.bath.ac.uk mailing list.

\section{Learning Objectives}
At the end of this part of the coursework you will be able to analyse, debug and extend medium-sized object oriented programs, drawing upon your knowledge of class design. inheritance and abstraction

\clearpage
\section{1. Codewalk}
In the first large coursework (SRPN), we gave you a piece of code to investigate and replicate. In this second coursework, we have given you a second piece of (broken) code that we would like you to analyse, repair and extend. As you consider this second piece of code, you will see that it is poorly designed, poorly commented and poorly written. Your high level objectives in this second piece of coursework are to write a report that

\begin{itemize}
\item describes the current shortcomings of the code
\item reports on the ways in which an understanding of Object Oriented Programming lead you to identify design improvements that might be made beyond the bug fixes introduced in the prevous item
\item  introduces the approach that you have taken to extending the current code such that it provides additional functionality in the following areas
\end{itemize}

\subsubsection*{2. Deliverables}
xx


\clearpage
\section{Assessment}
\subsection{Conditions}
The coursework will be conducted individually. Attention is drawn to the University rules on plagiarism. Whilst reference to exisitng code (with appropriate citation of that sourse) is permitted, we will only be able to assess your contribution.

\subsection{Marking}
Key issues for marking the code will be:
xxx

In cases where it is not clear from the assignment how it should be marked, you may be called on to explain or demonstrate your program. Such cases include but are not limited to suspected plagiarism

\section{Submission Instructions}
The deadline for this part of coursework is \textbf{5pm 24th November 2016}. Before the deadline upload your solution via Moodle.

Your submissions should be a zip file that can be extracted to a directory with the name Codewalk	-yourusername. The directory should contain.
xxx
 
Before you upload your solution, make sure that the zip file contains all necessary files and creates the correct directory. Please download the file from Moodle and double check that you have attached the right file, with the content that you want to be marked. You are responsible for checking that you are submitting the correct material to the correct assignment.

\section{Feedback}
Individual detailed feedback will be provided via Moodle within three weeks of the submission deadline. Note that marks will be moderated, this means that they become only final after the three weeks have expired. Clarification of the feedback can be obtained from the tutors during the labs.

\end{document}  