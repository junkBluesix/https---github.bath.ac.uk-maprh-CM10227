\documentclass[12pt, oneside]{article}   	% use "amsart" instead of "article" for AMSLaTeX format
\usepackage{geometry}                		% See geometry.pdf to learn the layout options. There are lots.
\geometry{a4paper}                   	% ... or a4paper or a5paper or ... 
%\geometry{landscape}                	% Activate for rotated page geometry
%\usepackage[parfill]{parskip}    	% Activate to begin paragraphs with an empty line rather than an indent
\usepackage{graphicx}		% Use pdf, png, jpg, or eps§ with pdflatex; use eps in DVI mode
					% TeX will automatically convert eps --> pdf in pdflatex		
\usepackage{amssymb}
\usepackage{listings}
\usepackage{enumerate}
\usepackage[T1]{fontenc}
\usepackage{lmodern}

\title{CM10277: Principles of Programming I \\ Large Coursework 2: Java Connect4}
\author{Dr. Rachid Hourizi \& Dr. Michael Wright \\ Dept. of Computer Science \\ University of Bath}
\date{}							% Activate to display a given date or no date

\begin{document}
\maketitle
\section{Introduction}
The coursework of this unit consists of three parts: the lab sheets and two larger courseworks. 

This document provides the specification for the second larger coursework: \textbf{Connect4}.

Questions regarding the coursework can always be posted on the Moodle Forum and programming1@lists.bath.ac.uk mailing list.

\section{Learning Objectives}
At the end of this coursework you will be able to 
\begin{itemize}
\item Plan, organise and implement program code to support reuse and maintainability of a software project.
\item Provide a critical review of a software in terms of software quality, design, reuse and robustness, and offer solutions to correct issues encountered.
\end{itemize}

\section{Connect4}
For your first coursework, we asked you to investigate and replicate a piece of code that we 
supplied (SRPN). In this second piece of coursework you will need to analyse, design and extend a second program that you will find on Moodle 
(Connect4).

The challenges in this coursework arise from the fact that the Java code supplied should allow 
a human to play the game Connect4  against a computer. Connect4 is described at 
https://en.wikipedia.org/wiki/Connect\_Four. Please ignore the ``Rule variations'' listed' on Wikipedia. You should only implement the standard version of the game
\bigskip
\bigskip

Unfortunately, our code
\begin{itemize}
\item doesn't compile or run,
\item is poorly designed (e.g. is largely made up of a huge main function without comments) and
\item  needs extension to meet an updated specification (see below).
\end{itemize}

To resolve these problems and, by extension, gain marks on this coursework, 

you will need to satisfy Requirements 1-5 below. 

\section{Requirements}
\begin{enumerate}
\item Provide a bug and ommission list (1-2 pages of single-spaced, 10 point font) explaining why our version of the code doesn't work (in the case of bugs) and/or doesnt provide the functionality needed to play Connect4. Each bug/ommission on the list should be described in the following format:
\begin{itemize}
\item Class : Line(s) : Bug/Omission : Type if Bug (syntax/run time/logic)
\item Solution.
\end{itemize}
e.g.
\begin{itemize}
\item Board.java : Line 7 : char entered where int expected : syntax error
\item Solution: Guard against non integer inputs.
\end{itemize}
\item Write a report (1-2 pages of single-spaced, 10 point font) describing the ways in which our code could be restructured to 
better reflect the fundamental OO concepts of modularisation and encapsulation. Include a discussion 
in that report of the extent to which you think that inheritance, abstract classes and interfaces 
might be used to improve our code. You may argue either for or against their inclusion.
\item Submit a revised version of our code which compiles and runs to provide a working 
version of the game Connect4. More specifically, provide an altered but 
uncompiled reworking of our code that a human marker can use to play a complete game of 
Connect4 against the computer. 
You may wish (but are not obliged) to start by commenting out large parts of our code and altering our printBoard() method. This approach has the advantage of giving you a relatively manageable starting point in the debugging process. You may also wish to tackle our placeCounter() method next.
\item Write a further report (1-2 pages of single-spaced, 10 point font) describing a program that allows
a human player to play 3-handed ConnectN against \textbf{two} computer players 
i.e. a game identical to Connect4 other than that 
\begin{itemize}
\item the winning condition is that N counters of the same colour are placed "in a row" and 
\item one human player competes against two computer players.
\end{itemize}
N will be passed to 
the code as a command line argument and \(2<N<7\). You may adopt an 
approach that keeps/replaces as much of our code as you find appropriate. You may include a single class diagram and up to half a page of high-level pseudo code in this part of your coursework.
\item Submit an updated and extended version of our code which compiles and runs to provide a working 
version of the game 3-handed ConnectN (as described in 4., above). More specifically, provide an ammended but 
uncompiled reworking of our code that a human marker can use to play a complete game of 
ConnectN against the computer.
\end{enumerate}

\section{Assessment}
50\% of the marks on CM50258 can be gained via coursework. The other 50\% can be gained in 
the exam.

The work described in this document can be used to gain up to 20\% of the course total 
(i.e. max 20/50 of the coursework mark). The remainder of the coursework marks can be 
gained via lab sheets (10\% max) and Large Coursework 1 (20\% max). 

You may not need to satisfy all parts of all 5 requirements, above, to pass the coursework 
portion of CM50258. Strategically, you may chose to concentrate on some rather than all 
of the requirements above. If you do so, we would suggest that you start by satisfying the 
'easier' requirements (1, 2 and 3). If you do so, however, be aware that this does 
not guarantee you a pass mark in the coursework portion of this course. If you have 
questions about this, or anything else in this document, please email Rachid or Michael.

\bigskip

The coursework will be conducted individually. Attention is drawn to the University rules on plagiarism. Whilst reference to exisitng code (with appropriate citation of that sourse) is permitted, we will only be able to assess your contribution.

You may be called on to explain or demonstrate your program. Cases leading to such a an explanation and/or demonstration include but are not limited to suspected plagiarism.


\section{Mark Scheme}
Marks for this coursework (Connect4) are available in the following areas:
\begin{enumerate}
\item Bug/Omission list: Max 20 marks.
\item Report on restructuring our code to reflect OO concepts: Max 20 marks.
\item Revised (i.e. working) Connect4 code: Max 20 marks.
\item Report on extending our code to support ConnectN: Max 20 marks.
\item Revised and extended ConnectN code: Max 20 marks i.e.
\begin {itemize}
\item 10 marks for  functionality and 
\item 10 marks for improved reflection of OO concepts.
\end{itemize} 
\end{enumerate}

\textbf{Total: Max 100 marks}.
\bigskip

For the code that you submit, marks will be awarded for functionality, code quality, appropriate use of object oriented programming principles and commenting.


\section{Submission Instructions}
The deadline for this part of coursework is \textbf{5pm 16th December 2016}. Before the deadline upload your solution via Moodle.

Your submissions should be a zip file that can be extracted to a directory with the name Connect4-yourusername. The directory should contain:
\begin{itemize}
\item a bug/omission list,
\item two reports,
\item two sub-directories containing the java files required to run Connect4 and 3-handed ConnectN respectively.
\end{itemize}
 
Before you upload your solution, make sure that the zip file contains all necessary files and sub-directories. Please download the file from Moodle and double check that you have everything that you want to be marked. You are responsible for checking that you are submitting the correct material to the correct assignment.

\section{Feedback}
Individual detailed feedback will be provided via Moodle within three weeks of the submission deadline. Note that marks will be moderated, this means that they become only final after the three weeks have expired. Clarification of the feedback can be obtained from the tutors during the labs.

\end{document}  