\documentclass{exam}
\title{CM10227: PROGRAMMING 1A}
\date{}
\examcode{CM10227}
\calculators
\papertype{PartII}

\usepackage{listings}
\usepackage{color}

\begin{document}
\maketitle

This paper contains questions designed to test both your understanding of core programming concepts and  your ability to demonstrate that understanding whilst writing short programs (or parts of programs). For the sake of clarity, you should note that the Java language should be used whenever a question asks you to either write code or provide examples. On that basis, wherever the paper asks you to provide one or more programs, classes, methods, statements, expressions, fields, etc. your answers should conform to the syntax rules of the Java language as taught in this year's CM10227 lectures (i.e. Java version 7).  

\clearpage





\

\begin{question}
\begin{roster}
\item[(a)]
A Java program is required to record information describing the coders, customer supporters and office managers working at a software company. Provide such a program to satisfy the requirements, below. 
\begin{itemize}
\item As you do so,  note that: 
\begin{itemize}
\item you need not provide a more complete implementation of the program than that which is explicitly specified  below.
\item for example, you do not need to write body code when providing methods
\item you should use at least one fully implemented \textit{class}, at least one \textit{interface} and at least one \textit{abstract class} in your answer. You may use more than one in each case.
\end{itemize}
\end{itemize}
\begin{itemize}
\item For each of the following roles, create a \textit{class}, \textit{interface} or \textit{abstract} class as appropriate: 
\begin{itemize}
\item employee, coder, team leader, software engineer, trainee, customer supporter, junior technician, senior technician, office manager.

\end{itemize}
\item Whilst creating these \textit{classes}, \textit{interfaces} and \textit{abstract classes}, also note that:
\begin{itemize}
\item everyone is an employee.
\item team leaders, software engineers and trainees are coders.
\item junior technicians and senior technicians are customer supporters.
\item team leaders and office managers supervise.
\item team leaders, software engineers and trainees write code 
\item junior technicians and senior technicians both log bugs but the 
\item process for logging bugs differs for each role
\end{itemize}
\item You should also add data \textit{fields} at appropriate points to capture the following information: 
\begin{itemize}
\item every employee has a name, address and employee number.
\item every coder has a specialist language. 
\item office managers have a spending limit
\end{itemize}
\bigskip
\item Finally, add methods for getting and setting the content of these fields. 

\end{itemize}
\marks{10}
\color{red}
\begin{lstlisting}
[1] for each of the following (max 5)
- Stub of Abstract CustomerSupporter class (or alternative Abstract Class)
- Correct extension of parent class to other subclasses
- Stub of `supervises' Interface
- Use of inheritance for 
  employees, coders and customer supporters (or sutiable alternative inheritance)
- Provision of at least one concrete class

[0.5] for each of the following defined (with constructors for classes):
- employee, coder, team leader, software engineer, trainee, customer supporter, junior technician, senior technician, office manager. (max 2.5)

[0.5] for setters and getters to handle
- name, address, employee number, 
specialist language, spending limit (max 2.5)
\end{lstlisting}
\color{black}

\item[(b)]
x
\marks{4}
\color{red}
\begin{lstlisting}
x
\end{lstlisting}
\color{black}

\item[(c)]
x
\marks{2}
\color{red}
\begin{lstlisting}
x
\end{lstlisting}
\color{black}

\item[(d)]
x
\marks{2}
\color{red}
\begin{lstlisting}
x
\end{lstlisting}
\color{black}

\item[(e)]
x
\marks{2}
\color{red}
\begin{lstlisting}
x
\end{lstlisting}
\color{black}

\item[(e)]
x
\marks{2}
\color{red}
\begin{lstlisting}
x
\end{lstlisting}
\color{black}
\end{roster}
\end{question}

\paperend{RH}
\end{document}
