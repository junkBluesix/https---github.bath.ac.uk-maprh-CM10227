\documentclass[12pt, a4paper, oneside]{article}

\usepackage[T1]{fontenc}
\usepackage[british]{babel}
\usepackage{lmodern}
\usepackage{hyperref}

\renewcommand{\familydefault}{\sfdefault}

\author{}
\date{}

\title{CM10227 Lab 4 (Optional)}

\newcounter{qstncntr}
\newcommand{\exerset}{ \vspace*{-11pt}\setcounter{qstncntr}{0} }
\newenvironment{exer}[1]{\vspace*{11pt}\noindent {{\bf Exercise} \stepcounter{qstncntr} \theqstncntr:} #1}{}

\begin{document}
\maketitle

This is an optional lab for those who wish to hone their programming skills.
The exercises are designed to be challenging and the lab is \textbf{not} assessed.
We would encourage you to ask questions on Moodle.

\section*{Exercises}
\setlength{\parindent}{0cm} %
\exerset

\begin{exer}
  Use brute force to find a solution to the following (i.e. try every possible number up to one hundred and stop if it works):
  In a sports contest, there were
  \begin{math}
    m
  \end{math}
 medals awarded on
 \begin{math}
   n
 \end{math}
successive days (\begin{math}
  n > 1
\end{math}).
 On the first day, one medal and
 \begin{math}
   \frac{1}{7}
 \end{math}
of the remaining
\begin{math}
  m - 1
\end{math}
medals were awarded.
On the second day, two medals and
\begin{math}
  \frac{1}{7}
\end{math}
of the now remaining medals were awarded; and so on. On the
\begin{math}
  n
\end{math}th and last day, the remaining
\begin{math}
  n
\end{math}
medals were awarded. How many days did the contest last, and how many medals were awarded altogether?
 [1967 IMO, Problem 6]
\end{exer}

\begin{exer}
  Determine all (there are two) three-digit numbers
  \begin{math}
    N
  \end{math}
  having the property that
  \begin{math}
    N
  \end{math}
 is divisible by
 \begin{math}
   11
 \end{math},
 and
 \begin{math}
   \frac{N}{11}
 \end{math}
 is equal to the sum of the squares of the digits of
 \begin{math}
   N
 \end{math}. [1960 IMO, problem 1]
\end{exer}

%pi
\begin{exer}
  One way to calculate
  \begin{math}
    \pi
  \end{math}
  is to test locations over a uniform grid of a
  \begin{math}
    2 \times 2
  \end{math}
  square containing a circle with radius
  \begin{math}
    1
  \end{math}.
  Use the distance from the centre to indicate whether or not the location lies in the circle.
  Simulate this and calculate
  \begin{math}
    \pi
  \end{math}
  as
  \begin{math}
    (2)(2)\frac{L_i}{L_i+L_o}
  \end{math} where
  \begin{math}
    L_i
  \end{math}
  is the number of locations inside the circle and
  \begin{math}
    L_o
  \end{math}
  is the number of locations outside the circle.
\end{exer}

\begin{exer}
  Find the smallest odd and non-prime number which cannot be written as the sum of a prime and twice a square:
  e.g.
  \begin{math}
    21 = 3 + 2\times3^2
  \end{math}.
  [Project Euler Problem 46]
\end{exer}

\begin{exer}
  A ``knight's tour'' is is a path through which a knight visits every square of a (international) chess board only once.
  Write a program to find a ``knight's tour'' for an 8 by 8 chess board.
  (Hint: use Warnsdorf's rule)
\end{exer}


\end{document}
