\documentclass[12pt, a4paper, oneside]{article}

\usepackage[T1]{fontenc}
\usepackage[british]{babel}
\usepackage{lmodern}
\usepackage{hyperref}

\renewcommand{\familydefault}{\sfdefault}

\author{}
\date{}

\title{CM10227 Lab2: Variables and Functions}

\newcounter{qstncntr}
\newcommand{\exerset}{ \vspace*{-11pt}\setcounter{qstncntr}{0} }
\newenvironment{exer}[1]{\vspace*{11pt}\noindent {{\bf Exercise} \stepcounter{qstncntr} \theqstncntr:} #1}{}

\begin{document}
\maketitle

Answer each question in a different file named \emph{Lab2ex<Number>.c }(e.g.\ Lab2ex1.c) (i.e. one file only per exercise).
We must be able to compile your code on \href{http://www.bath.ac.uk/guides/connecting-to-linux-bath/}{linux.bath} by typing \emph{gcc -o Lab2ex1 Lab2ex1.c -lm}, etc.
All your files should be submitted in a single zip, with all source (.c) files in one folder or the root of the zip:
the file name should be of the form \emph{<username>-lab<number>.zip}, e.g. \emph{abc12-lab2.zip}.

Your code will initially be marked by an automated process, followed by individual feedback and verification by a tutor.
Please, therefore, adhere to the above instructions and use the \textbf{exact format as given}.
Failure to do so will delay the marking process, and may incur penalties in the interest in providing everyone prompt feedback.
You are advised to re-download your Moodle submission and check it thoroughly, in advance of the deadline.

If you have questions about this lab sheet or are stuck with one of the questions talk to the tutors in the lab or post your question on Moodle.
Please do not include code on Moodle: bring this to a lab session.

This lab is part of your coursework.
\textbf{Please check Moodle for the submission deadline.}

\section*{Output format}
We will use an automated checker to ensure that your code meets the requirements defined above.
This means that the format of your outputs will matter.
The automated marking tool assumes that each result or sentence (exercise 4) is output on a separate line.

For example, if the five numbers in exercise 1 were one to five: the output must be as follows:
\begin{verbatim}
False
True
False
True
False
\end{verbatim}

\section*{Exercises}
\setlength{\parindent}{0cm} %
\exerset
\begin{exer}
Write a function called ``isEven'' that takes an integer as an argument and returns true if the argument is an even number and false if it is odd. Hint: Use ``\#include <stdbool.h>'' to obtain boolean type.
Using another function (or otherwise), print either ``True\textbackslash n'' or ``False\textbackslash  n'', for example, the output for arguments 3 and 42 would be as follows:
\begin{verbatim}
False
True
\end{verbatim}

Call this function with the following numbers: 10, 21, 33, 8, 200.
\end{exer}

\begin{exer}
Write a function that takes in an integer input between one and one hundred and prints out the number expressed in English text, with spaces and no dashes (--), e.g. for number ``33'', we would expect to see ``thirty three''.

Call this function with each of the following numbers: 10, 23, 100, 3, 30.
As before, each number should be on a new line. Hint: we recommend you call ``printf'' multiple times; we suggest you don't attempt to concatenate strings.
\end{exer}

\begin{exer}
Write a function that calculates the distance between two points.
The function should take, as an input, four arguments consisting of the two points' coordinates:
\begin{math}
  (x_1, y_1)
\end{math}
and
\begin{math}
  (x_2, y_2)
\end{math}.

Call the function with the following values, printing the result each time on a new line, to two decimal places:
\begin{enumerate}
\item (0,0) and (4,3)
\item  (21,3) and (7,9)
\item (-2,5) and (17,1)
\item  (-13,-5) and (9,2)
\item (-6,-4) and (-14,19)
\end{enumerate}
For example, if the numbers were (0, 0) and (3, 4), we would expect to see ``5.00'' printed.

Hint: you may wish to use the following print statement, or similar:
\begin{verbatim}
printf("%1.2f\n", calculateDistance(0, 0, 4, 3));
\end{verbatim}
\end{exer}

\begin{exer}
Write a function that is given an integer,
\begin{math}
  n
\end{math},
by the main method where
\begin{math}
  1 \leq n \leq 9999
\end{math}
and prints whether it is even, odd, or/and prime.
Call the function with the following numbers: 11, 74, 307, 7402, 9357.
Unlike other exercises, the output should be whole sentences.
Here is an example for the numbers one to five:
\begin{verbatim}
1 is odd and not prime.
2 is even and prime.
3 is odd and prime.
4 is even and not prime.
5 is odd and prime.
\end{verbatim}

Use multiple functions to enhance readability.
As before, we advise using multiple ``printf'' statements and not concatenating strings.

\end{exer}

\begin{exer}
 Write a function which multiples two integers together, but \textbf{without} the use of ``*'' or any iterative functions (including ``for'' or while loops). Instead the result must be calculated in a recursive fashion. Hint: use ``+'' and recursion.

Call the function five times from the main method with the following values, printing the results, one integer on each line e.g. ``10'' or ``-24'':
\begin{enumerate}
\item \begin{math}
    6\times3
  \end{math}

\item \begin{math}
  9\times9
\end{math}

\item \begin{math}
  3\times0
\end{math}

\item \begin{math}
  -1\times7
\end{math}

\item \begin{math}
  -29 \times 803
\end{math}
\end{enumerate}
\end{exer}

\end{document}
