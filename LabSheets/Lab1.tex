\documentclass[12pt, a4paper, oneside]{article}

\usepackage[T1]{fontenc}
\usepackage[british]{babel}
\usepackage{lmodern}
\usepackage{hyperref}

\renewcommand{\familydefault}{\sfdefault}

\author{}
\date{}

\title{CM10227 Lab1: Variables and Functions}

\newcounter{qstncntr}
\newcommand{\exerset}{ \vspace*{-11pt}\setcounter{qstncntr}{0} }
\newenvironment{exer}[1]{\vspace*{11pt}\noindent {{\bf Exercise} \stepcounter{qstncntr} \theqstncntr:} #1}{}

\begin{document}
\maketitle

Answer each question in a different file named \emph{Lab1ex<Number>.c }(e.g.\ Lab1ex1.c) (i.e. one file only per exercise).
We must be able to compile your code on \href{http://www.bath.ac.uk/guides/connecting-to-linux-bath/}{linux.bath} by typing \emph{gcc -o Lab1ex1 Lab1ex1.c -lm}, etc.
All your files should be submitted in a single zip, with all source (.c) files in one folder or the root of the zip:
the file name should be of the form \emph{<username>-lab<number>.zip}, e.g. \emph{abc12-lab1.zip}.

Your code will initially be marked by an automated process, followed by individual feedback and verification by a tutor.
Please, therefore, adhere to the above instructions and use the \textbf{exact format as given}.
Failure to do so will delay the marking process, and may incur penalties in the interest in providing everyone prompt feedback.
You are advised to re-download your Moodle submission and check it thoroughly, in advance of the deadline.

If you have questions about this lab sheet or are stuck with one of the questions talk to the tutors in the lab or post your question on Moodle.
Please do not include code on Moodle: bring this to a lab session.

This lab is part of your coursework.
\textbf{Please check Moodle for the submission deadline.}

\section*{Exercises}
\setlength{\parindent}{0cm} %
\exerset
\begin{exer}
  Adapt the ``HelloWorld'' code presented in the lecture to produce a program that defines a variable capable of holding an integer of your choice. The program should add 3 to that number, multiply the result by 2, subtract 4, subtract twice the original number, add 3, then print the result and a new line.
%You can test your code with (paste everything after \$ into the terminal) the following:
%\tiny
% \begin{verbatim}
%$ diff <(./Lab1ex1) <(printf "5\n") && echo Correct || echo Incorrect
%\end{verbatim}
\end{exer}

\begin{exer}
  Write a program which prints every integer from
  \begin{math}
    x
  \end{math}
to
\begin{math}
  x+10
\end{math},
 inclusive, for three different values of
 \begin{math}
   x
 \end{math}.
Do not use loops. Your program should do this with the following numbers: 5, 10 and 12, yielding the following output:
\begin{verbatim}
5
6
...
15
10
...
\end{verbatim}

%You can test the first 10 numbers using the following:
%\tiny
%\begin{verbatim}
% $ diff <(head -n 11 <(./Lab1ex2)) <(printf "5\n6\n7\n8\n9\n10\n11\n12\n13\n14\n15\n") && echo Correct || echo Incorrect
%\end{verbatim}
\end{exer}

\begin{exer}
  Write a program which converts the height of a person from centimetres to feet and inches.
Use integer division (rounding down is acceptable, which is the default for integer division).
Hint: 254 cm is exactly 100 inches and 12 inches is exactly 1 foot.
Perform five conversions as follows, and ensure your program outputs the following (which "?" replaced with the true value):

\begin{verbatim}
101 cm is 3 feet 3 inches to the nearest inch.
3 cm is 0 feet 1 inches to the nearest inch.
15 cm is ? feet ? inches to the nearest inch.
92 cm is ? feet ? inches to the nearest inch.
24 cm is ? feet ? inches to the nearest inch.
\end{verbatim}


%You can test the first two lines as follows:
%\tiny
%\begin{verbatim}
%$ diff <( head -n 2 <(./Lab1ex3)) <(printf "101 cm is 3 feet 3 inches to the nearest inch.\n3 cm is 0 feet 1 inches to the nearest inch.\n") && echo C || echo I
%\end{verbatim}
\end{exer}

\begin{exer}
 Write a program that uses three variables (current, previous, next) to calculate and print out the first ten numbers of the Fibonacci sequence, each on a new line: i.e. the first four lines should be as follows:
\begin{verbatim}
0
1
1
2
\end{verbatim}

%You can test the first four lines as follows:
%\tiny
%\begin{verbatim}
%$ diff <(head -n 4 <(./Lab1ex4)) <(printf "0\n1\n1\n2\n") && echo Correct || echo Incorrect
%\end{verbatim}
\end{exer}

\begin{exer}
  Create a program with two variables: height and radius (or use user defined functions). Use these two variables and print to the screen, the volume of three cylinders with the following dimensions:
  \begin{enumerate}
  \item height 7.0cm and radius 4.0cm
  \item height 20.0cm and radius 3.0cm
  \item height 14.7cm and radius 5.2cm
  \end{enumerate}
  Provide the result to two decimal places with in the format, e.g.\ the following (? replaced with real values):
\begin{verbatim}
The cylinder with height 7.0cm and radius 4.0cm has a volume of 351.86cm^3
The cylinder with height 20.0cm and radius 3.0cm has a volume of ?cm^3
The cylinder with height 14.7cm and radius 5.2cm has a volume of ?cm^3
\end{verbatim}

  If you use C math functions you will need to have \emph{\#include <math.h>} with \emph{\#include <stdio.h>}, and add \emph{-lm} as a compile flag, e.g.\ \emph{gcc -o Lab1ex5 Lab1ex5.c -lm}.

%You can test the first line as follows:
%
%{\tiny
%\begin{verbatim}
%$ diff <(head -n 1 <(./Lab1ex5)) <(printf "The cylinder with height 7.0cm and radius 4.0cm has a volume of 351.86cm^3\n") && echo Correct || echo Incorrect
%\end{verbatim}
%}


You may wish to use the following print statement:

\tiny
\begin{verbatim}
  printf("The cylinder with height %1.1fcm and radius %1.1fcm has a volume of %1.2fcm^3\n", height, radius, vol);
\end{verbatim}
\end{exer}

\end{document}
