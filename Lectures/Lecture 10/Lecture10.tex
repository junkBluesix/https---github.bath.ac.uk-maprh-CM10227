\documentclass{beamer}
% September 2014 
% Author: Dr Rachid Hourizi and Dr. Michael Wright 
% Department of Computer Science, University of Bath
\usepackage{listings}
\usetheme{Boadilla} 
\usepackage{fixltx2e}
\usepackage{hyperref}
\lstset{language=Java,,
	basicstyle=\ttfamily\small,
           keywordstyle=\color{blue}\ttfamily,
           stringstyle=\color{red}\ttfamily,
           commentstyle=\color{gray}\ttfamily,
          breaklines=true}

\begin{document}

\AtBeginSection[]{
  \begin{frame}
  \vfill
  \centering
  \begin{beamercolorbox}[sep=8pt,center,shadow=true,rounded=true]{title}
    \usebeamerfont{title}\insertsectionhead\par%
  \end{beamercolorbox}
  \vfill
  \end{frame}
}

\title{CM 10227: Lecture 10}
\author{Dr Rachid Hourizi and Dr. Michael Wright}
\date{\today}
\frame{\titlepage}

\begin{frame} 
\begin{center}
\textbf{Resources}
\end{center}
\begin{itemize}
\item More help with this course
\begin{itemize}
\item Moodle
\item E-mail - programming1@lists.bath.ac.uk
\end{itemize}
\item Online C and Java IDE
\begin{itemize}
\item https://www.codechef.com/ide
\item Remember to select Java from the drop down menu.
\end{itemize}
\item Books
\begin{itemize}
\item Java by Dissection (Free pdf online)
\item The Java Tutorial (https://docs.oracle.com/javase/tutorial/)
\end{itemize}
\end{itemize}
\end{frame}

\begin{frame} 
\begin{center}
\textbf{Resources}
\end{center}
\begin{itemize}
\item The places that you can get additional support if you are finding the pace of the course a little fast now include
\begin{itemize}
\item A labs (Continued from week 1)
\item B labs 
\item ... Wednesday 11:15-13:05 EB0.7
\item ... Fridays 17:15 to 19:15 in CB 5.13)
\item The Drop in Session 
\item ... booked 20 min appointments
\item ... Friday 11.15-13.05 1E 3.9
\item PAL sessions (Mondays 14:15 to 15:05 1E 3.9)
\end{itemize}
\end{itemize}
\end{frame}

\begin{frame}
\begin{center}
\textbf{Last week }
\end{center}
\begin{itemize}
\item Errors
\item Exceptions
\item Style : Writing Better Code
\end{itemize}
\end{frame}

\begin{frame}
\begin{center}
\textbf{Last week }
\end{center}
\begin{itemize}
\item Java API Libraries
\end{itemize}
\end{frame}

\begin{frame}
\begin{itemize}
\item As we have seen throughout the course, a large part of learning to program is learning how to re-use code e,g.
\begin{itemize}
\item Classes
\item Methods
\item Inheritance
\item Abstract Classes
\item Interfaces
\end{itemize}
\item As well as reusing our own code, we have also discussed opportunities to use those provided by the developers of the Java language
\end{itemize}
\end{frame}

\begin{frame}
\begin{itemize}
\item Java comes delivered with a library of useful classes, which are referred to as the Java API 
\item There really is no point in reinventing the wheel...
\end{itemize}
\end{frame} 

\begin{frame}
\begin{itemize}
\item Examples of Java library class include
\bigskip
\item Random - a library class used to generate random numbers
\item Math - a library class for performing basic numeric operations
\item ArrayList - a library class for storing and manipulating a resizable array
\item StringBuilder - a library class for manipulating a mutable sequence of characters
\end{itemize}
\end{frame}

\begin{frame} 
\begin{itemize}
\item The Java class library contains,
\begin{itemize}
\item Thousands of classes
\item Tens of thousands of methods
\item Many useful classes that make life much easier
\end{itemize}
\item These classes are organised in packages that follow a directory structure
\item A competent Java programmer must be able to work with the libraries.
\item (Just as competent programmers using other languages must be able to work with the libraries provided by those languages)
\end{itemize}
\end{frame} 

\begin{frame}
\begin{itemize}
\item We have looked at the specification of individual clases
\item Comprehensive documentation of the Java libraries is however available in HTML format; Readable in a web browser
\item Class API: Application Programmers Interface: 
\item Interface description for all library classes
\item \url{http://docs.oracle.com/javase/7/docs/api/}
\end{itemize}
\end{frame} 

\begin{frame}
\begin{itemize}
\item As you go forward with your Java programming, you should...
\bigskip
\item ... know some important classes by name; 
\item ... know how to find out about other classes.
\end{itemize}
\end{frame} 

\begin{frame}
\begin{itemize}
\item Remember! We only need to know the interface, not the implementation.
\bigskip
\item You will need to know about a library class because you will use objects of that class to provide common functionality within your programs
\item e.g. you may use objects of the Random class to generate random numbers
\end{itemize}
\end{frame} 

\begin{frame}[fragile]
\begin{itemize}
\item In this case, we use the class library in just the same way that we might use classes that we have created ourselves:
\item We create a new instance of the class (a new Object) using the new keyword
\begin{block}{}
\begin{lstlisting}
Random randomGenerator = new Random();
\end{lstlisting}
\end{block}
\item And can then use that Object's methods to provide the functionality that we need within our own program
\begin{block}{}
\begin{lstlisting}
int index2 = randomGenerator.nextInt(100);
\end{lstlisting}
\end{block}
\end{itemize}
\end{frame} 

\begin{frame}[fragile]
\begin{itemize}
\item Other useful constructor and methods in the Random class include
\begin{block}{}
\begin{lstlisting}
Random(long seed)
// seed is the initial value of the internal state of the pseudorandom number generator

public int nextInt()

public long nextLong()

public boolean nextBoolean()
\end{lstlisting}
\end{block}
\end{itemize}
\end{frame} 

\begin{frame}
\begin{itemize}
\item Another useful class is ArrayList
\bigskip
\item Resizable-array
\item Each ArrayList instance has a capacity and as elements are added its capacity grows automatically
\end{itemize}
\end{frame} 

\begin{frame}[fragile]
\begin{block}{}
\begin{lstlisting}
ArrayList<String> myArrayList 
    = new ArrayList<String>();

myArrayList.add("Hello");
myArrayList.add("World");
\end{lstlisting}
\end{block}
\end{frame} 

\begin{frame}[fragile]
\begin{itemize}
\item Other useful methods in the ArrayList class include
\begin{block}{}
\begin{lstlisting}
public boolean contains(Object o)

public E get(int index)

public int size()

public Object[] toArray()
\end{lstlisting}
\end{block}
\end{itemize}
\end{frame} 

\begin{frame}[fragile]
\begin{itemize}
\item Another really useful method in ArrayList is iterator()...
\item ... which returns an Iterator for the ArrayList
\end{itemize}
\begin{block}{}
\begin{lstlisting}
public Iterator<E> iterator()
\end{lstlisting}
\end{block}
\begin{itemize}
\item Iterator is an interface...
\item ... which defines a common set of methods to iterate over a collection
\end{itemize}
\end{frame} 

\begin{frame}[fragile]
\begin{itemize}
\item Why is this useful?
\item Lets think about generalisation and encapsulation of functionality
\item If we have a collection of elements we want to iterate over (i.e. loop through)...
\item ... do we really want to care about how find the size of the collection or get its next element?
\end{itemize}
\begin{block}{}
\begin{lstlisting}
for(int i=0; i<myArrayList.size(); i++){
    System.out.println(myArrayList.get(i));
}
\end{lstlisting}
\end{block}
\end{frame}

\begin{frame}
\begin{itemize}
\item Although most collections implement the AbstractList interface so have the size and get methods...
\item ... if these names were to change
\item ... or we want to ensure we use the fastest possible method of iteration over a collection
\item ... an Iterator object is preferable
\end{itemize}
\end{frame} 

\begin{frame}[fragile]
\begin{block}{}
\begin{lstlisting}
ArrayList<String> myArrayList 
    = new ArrayList<String>();

myArrayList.add("Hello");
myArrayList.add("World");

Iterator i = myArrayList.iterator();
        
while(i.hasNext()){
    System.out.print(i.next());
}
\end{lstlisting}
\end{block}

\begin{block}{}
\begin{lstlisting}
$ Hello World
\end{lstlisting}
\end{block}
\end{frame} 

\begin{frame}[fragile]
\begin{block}{}
\begin{lstlisting}
ArrayList<String> myArrayList 
    = new ArrayList<String>();

myArrayList.add("Hello");
myArrayList.add("World");

Iterator<String> i = myArrayList.iterator();
for( ; i.hasNext(); ){
    System.out.print(i.next());
}
\end{lstlisting}
\end{block}

\begin{block}{}
\begin{lstlisting}
$ Hello World
\end{lstlisting}
\end{block}
\end{frame} 

\begin{frame}
\begin{center}
\textbf{Aside: Generics}
\end{center}
\begin{itemize}
\item Generics allows you to specify types (classes and interfaces) to be parameters when defining classes, interfaces and methods
\bigskip
\item Advantages are...
\item ... stronger type checks at compile time
\item ... elimination of casts
\item ... enabling programmers to implement generic algorithms
\end{itemize}
\end{frame} 

\begin{frame}
\begin{center}
\textbf{Aside: Generics}
\end{center}
\begin{itemize}
\item Stronger type checks at compile time
\bigskip
\item Java compiler applies strong type checking to generic code 
\item Issues errors if the code violates type safety
\item Fixing compile-time errors is easier than fixing runtime errors
\end{itemize}
\end{frame} 

\begin{frame}[fragile]
\begin{center}
\textbf{Aside: Generics}
\end{center}
\begin{itemize}
\item Elimination of casts
\end{itemize}
\begin{block}{}
\begin{lstlisting}
List list = new ArrayList();
list.add("hello");
String s = (String) list.get(0);
\end{lstlisting}
\end{block}
\begin{block}{}
\begin{lstlisting}
List<String> list = new ArrayList<String>();
list.add("hello");
String s = list.get(0); // no cast
\end{lstlisting}
\end{block}
\end{frame} 

\begin{frame}[fragile]
\begin{center}
\textbf{Aside: Generics}
\end{center}
\begin{itemize}
\item Enabling programmers to implement generic algorithms
\end{itemize}
\begin{block}{}
\begin{lstlisting}
public class Box<T> {
    // T stands for "Type"
    private T t;

    public void set(T t) { this.t = t; }
    public T get() { return t; }
}
\end{lstlisting}
\end{block}
\begin{block}{}
\begin{lstlisting}
Box<Integer> box1 = new Box<Integer>();
Box<String> box1 = new Box<String>();
\end{lstlisting}
\end{block}
\end{frame} 

\begin{frame}[fragile]
\begin{center}
\textbf{Aside: Generics}
\end{center}
\begin{itemize}
\item The most commonly used type parameter names are...
\end{itemize}
\begin{block}{}
\begin{lstlisting}
E - Element (used extensively by the Java Collections Framework)
K - Key
N - Number
T - Type
V - Value
\end{lstlisting}
\end{block}
\end{frame} 

\begin{frame}
\begin{center}
\textbf{Back to Java API}
\end{center}
\begin{itemize}
\item Interface vs. Implementation
\item For each class provided by the Java API, the Java developers provide documentation
\item The documentation includes:

\begin{itemize}
\item the name of the class;
\item a general description of the class;
\item a list of constructors and methods
\item return values and parameters for constructors and methods
\item a description of the purpose of each constructor and method
\end{itemize}
\item The interface of the class
\end{itemize}
\end{frame} 

\begin{frame}
\begin{itemize}
\item The documentation does not include
\item Private fields 
\begin{itemize}
\item (most fields are private) 
\end{itemize}
\item Private methods
\item The bodies (implementation code) for each method
\item Detailed implementation of the class
\end{itemize}
\end{frame} 

\begin{frame}
\begin{itemize}
\item Most classes from the library must be imported using an import statement 
\item (Those from java.lang, however, do not)
\item They can then be used like classes from the current project.
\end{itemize}
\end{frame} 

\begin{frame}[fragile]
\begin{itemize}
\item Classes are organised in packages. 
\item Single classes may be imported:
\item Whole packages can be imported:
\end{itemize}

\begin{block}{}
\begin{lstlisting}
import java.util.ArrayList;

import java.util.*;
\end{lstlisting}
\end{block}
\end{frame} 

\begin{frame}[fragile]
\begin{center}
\textbf{Another Example of the Java API}
\end{center}
\begin{itemize}
\item StringBuilder
\bigskip
\item Lets consider string concatenation 
\end{itemize}

\begin{block}{}
\begin{lstlisting}
String name = "Michael";
String lastName = "Wright";

System.out.println(name + " " + lastName);
\end{lstlisting}
\end{block}
\end{frame}

\begin{frame}[fragile]
\begin{itemize}
\item Remember that String is immutable
\item So when we concatenate a new String is created
\item For this example it is not a problem
\item i.e. the ``performance'' of the program is not likely to suffer
\end{itemize}

\begin{block}{}
\begin{lstlisting}
String name = "Michael";
String lastName = "Wright";

System.out.println(name + " " + lastName);
\end{lstlisting}
\end{block}
\end{frame}

\begin{frame}[fragile]
\begin{itemize}
\item But what about this...
\end{itemize}
\begin{block}{}
\begin{lstlisting}
String s = "";
for(int i=0; i<50000; i++){
    s = s + " " + i;
}
\end{lstlisting}
\end{block}
\end{frame}

\begin{frame}
\begin{itemize}
\item For large amounts of concatenation, especially in loops...
\item ... we might consider using a StringBuilder
\bigskip
\item A mutable sequence of characters
\item Includes method to append, insert and reverse
\item toString() is also a useful method
\end{itemize}
\end{frame}

\begin{frame}[fragile]
\begin{itemize}
\item StringBuilder example
\end{itemize}
\begin{block}{}
\begin{lstlisting}
StringBuilder s = new StringBuilder();

for(int i=0; i<50000; i++){
    s.append(" ");
    s.append(i);        
}
\end{lstlisting}
\end{block}
\end{frame}

\begin{frame}[fragile]
\begin{itemize}
\item Different ways to concatenate a String
\end{itemize}
\begin{block}{}
\begin{lstlisting}
// +
String fullName = "Michael" + " " + "Wright";

// concat method in String
String fullName = "Michael".concat(" Wright");

// StringBuilder
StringBuilder fullName = new StringBuilder();
fullName.append("Michael");
fullName.append(" ");
fullName.append("Wright");
\end{lstlisting}
\end{block}
\end{frame}

\section{Class Variable and Methods}
\begin{frame}
\begin{itemize}
\item Class libraries can also define constants
\item A constant is an identifier (a name) with an associated value which cannot be altered by the program during normal execution -- the value is constant. 
\item This is contrasted with a variable, which is an identifier (a name) associated with a value that can be changed during normal execution -- the value is variable.
\end{itemize}
\end{frame} 

\begin{frame}
\begin{itemize}
\item Math.PI (the number pi) is and example of a constant 
\item We only need to store pi once 
\item And can then use it over and over again in our programs
\item If we look at the documentation for Math.PI in the Java API 
\item (i.e. the documentation for the field PI within the Math class)
\item We can see it is a static final i.e. it is constant and stored at the class level
\item \url{http://docs.oracle.com/javase/7/docs/api/java/lang/Math.html}
\end{itemize}
\end{frame} 

\begin{frame}
\begin{itemize}
\item Static methods
\bigskip
\item Note that it is also possible to create class methods
\item i.e. methods that are declared as static
\item These methods can be called directly from the class rather than from an Object that instantiates that class
\end{itemize} 
\end{frame}


\begin{frame}[fragile]
\begin{itemize}
\item The Math class, for example, provides a method that raises the first argument (a) to the power of the second argument (b):
\end{itemize} 
\begin{block}{}
\begin{lstlisting}
public static double pow(double a, double b);
\end{lstlisting}
\end{block}
\end{frame} 

\begin{frame}[fragile]
\begin{block}{}
\begin{lstlisting}
public class MyClass{

  public void printVolume(double radius){
    double volume = Math.pow(Math.PI*radius, 2);
    System.out.println("volume is: " +volume);
  } 
  
}
\end{lstlisting}
\end{block}
\end{frame}

\begin{frame}[fragile]
\begin{itemize}
\item The Double class provides a method that parses a String into an double:
\end{itemize} 
\begin{block}{}
\begin{lstlisting}
public static double parseDouble(String s);
\end{lstlisting}
\end{block}
\end{frame} 

\begin{frame}[fragile]
\begin{block}{}
\begin{lstlisting}
public class MyClass{

  public void printVolume(String radiusString){
    double radius = Double.parseDouble(radiusString);
    
    double volume = Math.pow(Math.PI*radius, 2);
    System.out.println("volume is: " +volume);
  } 
  
}
\end{lstlisting}
\end{block}
\end{frame}

\begin{frame}[fragile]
\begin{itemize}
\item Note that we can create our own
\begin{itemize}
\item constants
\item class variables
\item and class constants
\end{itemize} 
\item We can also create our own class methods
\end{itemize} 
\begin{block}{}
\begin{lstlisting}
final int myConstant = 1;
static int myClassVariable = 2;
static final int myClassConstant = 3;

public static void myClassMethod(){
    ...
} 
\end{lstlisting}
\end{block}
\end{frame} 

\begin{frame}[fragile]
\begin{itemize}
\item \textbf{Instance methods ... }
\item ... are associated with an object and 
\item ... use the instance variables of that object. 
\item This is the default. 
\end{itemize} 
\begin{block}{}
\begin{lstlisting}
public class MyClass{
    
    private int myVariable;
    
    public int getMyVariable(){
        return myVariable;
    }

}
\end{lstlisting}
\end{block}
\end{frame} 

\begin{frame}
\begin{itemize}
\item \textbf{Static methods ... }
\item ... use no instance variables of any object of the class they are defined in
\item ... (if you define a method to be static, you will be given a rude message by the compiler if you try to access any instance variables) 
\item ... you can access static variables, but except for constants, this is unusual. 
\item ... typically take all they data from parameters and compute something from those parameters, with no reference to variables. 
\item ... this is typical of methods which do some kind of generic calculation
\end{itemize} 
\end{frame} 

\begin{frame}[fragile]
\begin{block}{}
\begin{lstlisting}
public class MyUtilClass{
    
    public static double mean(int[] p) {
        int sum = 0;  // sum of all the elements
        for (int i=0; i<p.length; i++) {
            sum += p[i];
        }
        return ((double)sum) / p.length;
    }

}
\end{lstlisting}
\end{block}
\end{frame} 

\begin{frame}[fragile]
\begin{itemize}
\item Other examples of static methods in the Java API
\end{itemize} 
\begin{block}{}
\begin{lstlisting}
java.awt.Color

// static variables
Color.BLACK
Color.RED

// static methods
public static Color getColor(String nm)

public static int HSBtoRGB(float hue,
           float saturation,
           float brightness)
\end{lstlisting}
\end{block}
\end{frame} 

\begin{frame}
\begin{center}
\textbf{Summary}
\end{center}
\begin{itemize}
\item Java comes delivered with a library of useful classes, which are referred to as the Java API 
\item There really is no point in reinventing the wheel...
\end{itemize}
\end{frame} 

\begin{frame}
\begin{itemize}
\item As you go forward with your Java programming, you should...
\bigskip
\item ... know some important classes by name
\item ... know how to find out about other classes
\end{itemize}
\end{frame} 

\end{document}

