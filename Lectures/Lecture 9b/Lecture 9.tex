\documentclass{beamer}
% September 2014 
% Author: Dr Rachid Hourizi and Dr. Michael Wright 
% Department of Computer Science, University of Bath
\usepackage{listings}
\usetheme{Boadilla} 
\usepackage{fixltx2e}
\usepackage{hyperref}
\lstset{language=Java,,
	basicstyle=\ttfamily\small,
           keywordstyle=\color{blue}\ttfamily,
           stringstyle=\color{red}\ttfamily,
           commentstyle=\color{gray}\ttfamily,
          breaklines=true}

\begin{document}

\AtBeginSection[]{
  \begin{frame}
  \vfill
  \centering
  \begin{beamercolorbox}[sep=8pt,center,shadow=true,rounded=true]{title}
    \usebeamerfont{title}\insertsectionhead\par%
  \end{beamercolorbox}
  \vfill
  \end{frame}
}

\title{CM 10227: Lecture 9}
\author{Dr Rachid Hourizi and Dr Michael Wright}
\date{\today}
\frame{\titlepage}

\begin{frame} 
\begin{center}
\textbf{Resources}
\end{center}
\begin{itemize}
\item More help with this course
\begin{itemize}
\item Moodle
\item E-mail - programming1@lists.bath.ac.uk
\end{itemize}
\item Online C and Java IDE
\begin{itemize}
\item https://www.codechef.com/ide
\item Remember to select Java from the drop down menu.
\end{itemize}
\item Books
\begin{itemize}
\item Java by Dissection (Free pdf online)
\item The Java Tutorial (https://docs.oracle.com/javase/tutorial/)
\end{itemize}
\end{itemize}
\end{frame}

\begin{frame} 
\begin{center}
\textbf{Resources}
\end{center}
\begin{itemize}
\item The places that you can get additional support if you are finding the pace of the course a little fast now include
\begin{itemize}
\item A labs (Continued from week 1)
\item B labs 
\item ... Wednesday 11:15-13:05 EB0.7
\item ... Fridays 17:15 to 19:15 in CB 5.13)
\item The Drop in Session 
\item ... booked 20 min appointments
\item ... Friday 11.15-13.05 1E 3.9
\item PAL sessions (Mondays 14:15 to 15:05 1E 3.9)
\end{itemize}
\end{itemize}
\end{frame}

\begin{frame}
\begin{center}
\textbf{This week }
\end{center}
\begin{itemize}
\item Errors
\item Exceptions
\item Style : Writing Better Code
\end{itemize}
\end{frame}

\section{Style : Writing Better Code}

\begin{frame}
\begin{center}
\textbf{Reading from a File}
\end{center}
\begin{itemize}
\item We know how to read from the standard in...
\item ... using the Scanner class
\item ... using the BufferedReader class
\end{itemize}
\end{frame}

\begin{frame}[fragile]
\begin{center}
\textbf{BufferedReader : To Read From Standard In}
\end{center}
\begin{block}{}
\begin{lstlisting}
BufferedReader input = new BufferedReader(new InputStreamReader(System.in));

try {
    String userInput = input.readLine();
    ...
}
catch (IOException e) {
    e.printStackTrace();
}
\end{lstlisting}
\end{block}
\end{frame}

\begin{frame}[fragile]
\begin{center}
\textbf{InputStreamReader}
\end{center}
\begin{itemize}
\item From the Java Doc...
\bigskip
\item An InputStreamReader is a bridge from byte streams to character streams 
\item It reads bytes and decodes them into characters using a specified charset
\item The charset that it uses may be specified by name.. 
\item ... or may be given explicitly 
\item ... or the platform's default charset may be accepted
\bigskip
\item Subclass of Reader which is an abstract class for reading character streams
\end{itemize}
\end{frame}

\begin{frame}[fragile]
\begin{center}
\textbf{BufferedReader}
\end{center}
\begin{itemize}
\item From the Java Doc...
\bigskip
\item Reads text from a character-input stream...
\item ... buffering characters so as to provide for the efficient reading of characters, arrays, and lines
\end{itemize}
\end{frame}

\begin{frame}[fragile]
\begin{center}
\textbf{BufferedReader}
\end{center}
\begin{itemize}
\item So if BufferedReader reads text from a character-input stream...
\item ... and we know its constructor takes an object of type Reader as it parameter
\item ... and we know that InputStreamReader is a subclass of Reader
\item ... and we know that another subclass of Reader is FileReader
\item ... then (hopefully) the following code is not too surprising 
\end{itemize}
\end{frame}

\begin{frame}[fragile]
\begin{center}
\textbf{BufferedReader : To Read From a File}
\end{center}
\begin{block}{}
\begin{lstlisting}
File myFile = new File("example.txt");
FileReader fileIn = new FileReader(myFile);
BufferedReader input = new BufferedReader(fileIn);

try {
    String line = input.readLine();
    ...
}
catch (IOException e) {
    e.printStackTrace();
}
\end{lstlisting}
\end{block}
\end{frame}

\section{Style : Writing Better Code}

\begin{frame}
\begin{center}
\textbf{Style : Writing Better Code}
\end{center}
\begin{itemize}
\item Maintainance 
\item Coupling - Cohesion 
\item Improving Code 
\end{itemize}
\end{frame}

\begin{frame}
\begin{center}
\textbf{Software Changes}
\end{center}
\begin{itemize}
\item Software is not like a novel that is written once and then remains unchanged.
\item Software is extended, corrected, maintained, ported, adapted
\item The work is done by different people over time (often decades).
\end{itemize}
\end{frame}

\begin{frame}
\begin{itemize}
\item Change or Become Useless!
\item There are only two options for software: 
\begin{itemize}
\item Either it is continuously maintained
\item Or it becomes useless.
\end{itemize}
\item Software that cannot be maintained will be thrown away.
\end{itemize}
\end{frame}

\begin{frame}
\begin{center}
\textbf{Code Quality}
\end{center}
\begin{itemize}
\item Two important concepts for quality of evolving code:
\bigskip
\item Coupling 
\item Cohesion
\end{itemize}
\end{frame}

\begin{frame}
\begin{center}
\textbf{Coupling}
\end{center}
\begin{itemize}
\item Coupling refers to links between separate units of a program.
\item If two classes depend closely on many details of each other, we say they are tightly coupled.
\item Aim for loose coupling.
\end{itemize}
\end{frame}

\begin{frame}
\begin{center}
\textbf{Loose Coupling}
\end{center}
\begin{itemize}
\item Loose coupling makes it possible to:
\begin{itemize}
\item understand one class without reading others
\item change one class without affecting others
\end{itemize}
\item Thus: improves maintainability.
\end{itemize}
\end{frame}

\begin{frame}
\begin{center}
\textbf{Cohesion}
\end{center}
\begin{itemize}
\item Cohesion refers to the the number and diversity of tasks that a single unit is responsible for.
\item If each unit is responsible for one single logical task, we say it has high cohesion.
\item Cohesion applies to classes and methods. 
\item Aim for high cohesion.
\end{itemize}
\end{frame}

\begin{frame}
\begin{center}
\textbf{High Cohesion}
\end{center}
\begin{itemize}
\item High cohesion makes it easier to:
\bigskip
\item Understand what a class or method does
\item Use descriptive names
\item Reuse classes or methods
\end{itemize}
\end{frame}

\begin{frame}
\begin{center}
\textbf{Cohesion of Methods}
\end{center}
\begin{itemize}
\item A method should be responsible for one and only one well defined task
\end{itemize}
\end{frame}

\begin{frame}
\begin{center}
\textbf{Cohesion of Classes}
\end{center}
\begin{itemize}
\item Classes should represent one single, well defined entity.
\end{itemize}
\end{frame}

\begin{frame}
\begin{center}
\textbf{Design Questions}
\end{center}
\begin{itemize}
\item Common questions
\begin{itemize}
\item How long should a class be?
\item How long should a method be?
\end{itemize}
\item Can now be answered in terms of cohesion and coupling
\end{itemize}
\end{frame}

\begin{frame}
\begin{center}
\textbf{Design Guidelines}
\end{center}
\begin{itemize}
\item A method is too long if it does more then one logical task.
\item A class is too complex if it represents more than one logical entity.
\item Note these are guidelines. They still leave much open to the designer
% \item We will look at these guidelines in more detail in this afternoon's lecture
\end{itemize}
\end{frame}

\begin{frame}
\begin{center}
\textbf{Code Duplication}
\end{center}
\begin{itemize}
\item Is an indicator of bad design,
\item Makes maintenance harder,
\item Can lead to introduction of errors during maintenance.
\end{itemize}
\end{frame}

\begin{frame}
\begin{center}
\textbf{Responsibility-Driven Design}
\end{center}
\begin{itemize}
\item Question: where should we add a new method (which class)?
\item Each class should be responsible for manipulating its own data.
\item The class that owns the data should be responsible for processing it.
\item Responsibility-driven design leads to low coupling.
\end{itemize}
\end{frame}

\begin{frame}
\begin{center}
\textbf{Localising Change}
\end{center}
\begin{itemize}
\item One aim of reducing coupling and responsibility-driven design is to localize change.
\item When a change is needed, as few classes as possible should be affected.
\end{itemize}
\end{frame}

\begin{frame}
\begin{center}
\textbf{Thinking Ahead}
\end{center}
\begin{itemize}
\item When designing a class, we try to think what changes are likely to be made in the future.
\item We aim to make those changes easy.
\end{itemize}
\end{frame}

\begin{frame}
\begin{center}
\textbf{Refactoring}
\end{center}
\begin{itemize}
\item When classes are maintained, often code is added. 
\item Classes and methods tend to become longer.
\item Every now and then, classes and methods should be refactored to maintain cohesion and low coupling.
\end{itemize}
\end{frame}

\begin{frame}
\begin{center}
\textbf{Refactoring and Testing}
\end{center}
\begin{itemize}
\item When refactoring code, separate the refactoring from making other changes.
\item First do the refactoring only, without changing the functionality.
\item Test before and after refactoring to ensure that nothing was broken.
\end{itemize}
\end{frame}

\end{document}