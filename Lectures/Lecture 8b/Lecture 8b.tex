\documentclass{beamer}
% September 2014 
% Author: Dr Rachid Hourizi and Dr. Michael Wright 
% Department of Computer Science, University of Bath
\usepackage{listings}
\usetheme{Boadilla} 
\usepackage{fixltx2e}
\usepackage{hyperref}
\lstset{language=Java,,
	basicstyle=\ttfamily\small,
           keywordstyle=\color{blue}\ttfamily,
           stringstyle=\color{red}\ttfamily,
           commentstyle=\color{gray}\ttfamily,
          breaklines=true}

\begin{document}

\AtBeginSection[]{
  \begin{frame}
  \vfill
  \centering
  \begin{beamercolorbox}[sep=8pt,center,shadow=true,rounded=true]{title}
    \usebeamerfont{title}\insertsectionhead\par%
  \end{beamercolorbox}
  \vfill
  \end{frame}
}

\title{CM 10227: Lecture 8}
\author{Dr Rachid Hourizi and Dr Michael Wright}
\date{\today}
\frame{\titlepage}

\begin{frame} 
\begin{center}
\textbf{Resources}
\end{center}
\begin{itemize}
\item More help with this course
\begin{itemize}
\item Moodle
\item E-mail - programming1@lists.bath.ac.uk
\end{itemize}
\item Online C and Java IDE
\begin{itemize}
\item https://www.codechef.com/ide
\item Remember to select Java from the drop down menu.
\end{itemize}
\item Books
\begin{itemize}
\item Java by Dissection (Free pdf online)
\item The Java Tutorial (https://docs.oracle.com/javase/tutorial/)
\end{itemize}
\end{itemize}
\end{frame}

\begin{frame} 
\begin{center}
\textbf{Resources}
\end{center}
\begin{itemize}
\item The places that you can get additional support if you are finding the pace of the course a little fast now include
\begin{itemize}
\item A labs (Continued from week 1)
\item B labs 
\item ... Wednesday 11:15-13:05 EB0.7
\item ... Fridays 17:15 to 19:15 in CB 5.13)
\item The Drop in Session 
\item ... booked 20 min appointments
\item ... Friday 11.15-13.05 1E 3.9
\item PAL sessions (Mondays 14:15 to 15:05 1E 3.9)
\end{itemize}
\end{itemize}
\end{frame}

% *** CONTENT ***

\begin{frame}
\begin{center}
\textbf{This Week}
\end{center} 
\begin{itemize}
\item Interfaces 
\item Abstract Classes
\bigskip
\item Choosing to extend a Class or Abstract Class ...
\item ... or implement an Interface
\end{itemize}
\end{frame}

\begin{frame}
\begin{center}
\textbf{Recap: Interfaces and Abstract Classes}
\end{center}
\end{frame}

\begin{frame}
\begin{center}
\textbf{Interfaces}
\end{center}
\end{frame}

% *** INTERFACES ***

\begin{frame}
\begin{itemize}
\item At some point we may want an even looser relationship between superclass and subclass
\item We may want to specify a superclass in which no methods are implemented...
\item ... leaving subclasses to do all of the implementation
\end{itemize}
\end{frame}

\begin{frame}
\begin{itemize}
\item A Java Interface is a specification of a type 
\item ... in the form of a type name and methods 
\item ... that does not define any implementation of methods
\bigskip
\item Interfaces are more properly described as \textbf{design patterns}
\end{itemize}
\end{frame}

\begin{frame}[fragile]
\begin{itemize}
\item Animal interface
\end{itemize}
\begin{block}{}
\begin{lstlisting}
interface Animal{
   public static final int CONSTANT_VARIABLE = 42;
   
   String makeSound();

}
\end{lstlisting}
\end{block}
\end{frame}

\begin{frame}[fragile]
\begin{itemize}
\item Classes implement an interface 
\end{itemize}
\begin{block}{}
\begin{lstlisting}

public class Fox implements Animal{
    public String makeSound(){
    	return "Ring-ding-ding-ding-dingeringeding!";
    }
}
\end{lstlisting}
\end{block}
\end{frame}

\begin{frame}[fragile]
\begin{itemize}
\item NEW : Abstract Classes can also implement an interface 
\end{itemize}
\begin{block}{}
\begin{lstlisting}

abstract class Dog implements Animal { 
	public String makeSound(){
	    return "woof";
	}
	
	abstract String breedStandard();
}

  OR

abstract class Dog implements Animal { 
	abstract String breedStandard();
}
\end{lstlisting}
\end{block}
\end{frame}

\begin{frame}
\begin{itemize}
\item Important to note that Interfaces are not classes
\item ... they can not be instantiated 
\item ... they cannot share a name with a class
\item ... cannot contain implemented methods 
\item ... can only contain method stubs and constants
\end{itemize}
\end{frame}

\begin{frame}[fragile]
\begin{itemize}
\item Classes implementing an Interface do not inherit code, but ...
\item ... are subtypes of the interface type.
\item So, polymorphism is available with interfaces as well as classes.
\end{itemize}
\begin{block}{}
\begin{lstlisting}
Animal fox = new Fox();

Item item = new MusicFile();
Item item = new VideoFile();
\end{lstlisting}
\end{block}
\end{frame}

\begin{frame}[fragile]
\begin{itemize}
\item Why does this produce and error?
\end{itemize}
\begin{block}{}
\begin{lstlisting}
Animal dog = new FrenchBulldog();

String sound = dog.makeSound();
String description = dog.breedDescription(); // ERROR
\end{lstlisting}
\end{block}
\end{frame}

% *** ABSTRACT CLASSES ***
\begin{frame}
\begin{center}
\textbf{Abstract Classes}
\end{center}
\end{frame}

\begin{frame}
\begin{itemize}
\item In some situations, however, we may want to include methods in superclasses but only allow them to be used by subclasses
\item For example, in the Animal class we might want to include common functionality for all animals 
\item ... such as average life span
\end{itemize}
\end{frame}

\begin{frame}[fragile]
\begin{itemize}
\item ... we may also want to include an makeSound() method in the Animal class 
\item ... but only allow it to be used in classes that describe specific kinds of animal (i.e. in subclasses of Animal)
\bigskip
\item In these cases, we can use abstract classes and methods
\end{itemize}
\end{frame}

\begin{frame}[fragile]
\begin{block}{}
\begin{lstlisting}
public abstract class Animal{
   private int averageAge;
   
   public Animal(int averageAge){
       this.averageAge = averageAge;
   }

   public int getAverageAge(){
      return averageAge;
   }
   
   // more methods 
   
   // Make the sound of this animal
   abstract String makeSound();  
}
\end{lstlisting}
\end{block}
\end{frame}

\begin{frame}[fragile]
\begin{block}{}
\begin{lstlisting}
public class Dog extends Animal{

   private String sound;
   
   public Dog(int averageAge, String sound){
       super(averageAge);
       this.sound = sound;
   }
   
   // Make the sound of this animal
   public String makeSound(){
       return sound;
   }  
   
}
\end{lstlisting}
\end{block}
\end{frame}

\begin{frame}[fragile]
\begin{itemize}
\item NEW : Abstract Classes can also extend Abstract Classes
\end{itemize}
\begin{block}{}
\begin{lstlisting}
public abstract class Dog extends Animal{
   public static final String SOUND = "woof";
   
   public Dog(int averageAge){
       super(averageAge);
   }
   
   // Make the sound of this animal
   public String  makeSound(){
   	return sound;
   }
   
   abstract String breedStandard();
   
}
\end{lstlisting}
\end{block}
\end{frame}

\begin{frame}[fragile]
\begin{itemize}
\item NEW : Abstract Classes can also extend Abstract Classes
\end{itemize}
\begin{block}{}
\begin{lstlisting}
public class BoxerDog extends Dog{
   
   private String breedStandard; 
   
   public BoxerDog(int averageAge, String sound, String breedStandard){
       super(averageAge, sound);
       this.breedStandard = breedStandard;
   }
   
   public String breedStandard(){
       return breedStandard;
   }
}
\end{lstlisting}
\end{block}
\end{frame}

\begin{frame}
\begin{itemize}
\item Note that some methods can still be written in full in an abstract class \ ('implemented{}' methods)
\bigskip
\item So an abstract class can have 
\begin{itemize}
\item constants
\item fields
\item abstract methods
\item implemented methods
\end{itemize}
\end{itemize}
\end{frame}

\begin{frame}
\begin{itemize}
\item Also note that, like Interfaces, Abstract Classes can not be instantiated
\item Classes implementing an Abstract Class are subtypes of the Abstract Class type.
\item So, polymorphism is available with Abstract Class as well as classes.
\end{itemize}
\end{frame}

\begin{frame}
\begin{itemize}
\item Note, however, that Abstract Classes are still classes
\item ... more specifically, partially implemented classes
\bigskip
\item So the rules that govern implemented classes also govern abstract classes 
\item i.e. no subclass can have two (or more) abstract parents
\end{itemize}
\end{frame}

\begin{frame}
\begin{center}
\textbf{Bringing this all together...}
\end{center}
\begin{itemize}
\item For example, 
\bigskip
\item ... consider methods of payment accepted in your application
\item ... we know we need to take payment but there are different ways of paying
\item ... e.g. Credit Card, PayPal etc..
\end{itemize}
\end{frame}

\begin{frame}[fragile]
\begin{block}{}
\begin{lstlisting}
interface Payment{
    void makePayment(double debit);
}
\end{lstlisting}
\end{block}
\end{frame}

\begin{frame}[fragile]
\begin{block}{}
\begin{lstlisting}
public class PayPal implements Payment{
    public void makePayment(double debit) {
        // logic for PayPal payment
        // e.g. Paypal uses username and password for payment
    }
}
public class CreditCard implements Payment{
    public void makePayment(double debit){
        // logic for CreditCard payment
        // e.g. CreditCard uses card number, date of expiry etc...
    }
}
\end{lstlisting}
\end{block}
\end{frame}

\begin{frame}
\begin{center}
\textbf{Bringing this all together...}
\end{center}
\begin{itemize}
\item BUT, if for example... 
\item ... each payment type (PayPal, Credit Card etc.) needs to be authorised by a Bank
\item ... and this process is the same for all types of payment
\item ... and we don't want programmers to reimplement this each time
\bigskip
\item ... we would use an Abstract Class to implement this method
\item ... but ensure that programmers implement the makePayment method
\item ... because this is specific to each payment type
\end{itemize}
\end{frame}

\begin{frame}[fragile]
\begin{block}{}
\begin{lstlisting}
public abstract class Payment{

    protected boolean authoriseWithBank(String userAuthKey){
        // logic to authorise payment with bank
    }

    abstract void makePayment(double debit);
}
\end{lstlisting}
\end{block}
\end{frame}

\begin{frame}[fragile]
\begin{block}{}
\begin{lstlisting}
public class PayPal extends Payment{
    public void makePayment(double debit) {
        // logic for PayPal payment
        // e.g. Paypal uses username and password for payment
        super.authoriseWithBank(authToken);
    }
}
public class CreditCard extends Payment{
    public void makePayment(double debit){
        // logic for CreditCard payment
        // e.g. CreditCard uses card number, date of expiry etc...
        super.authoriseWithBank(authToken);
    }
}
\end{lstlisting}
\end{block}
\end{frame}

\begin{frame}
\begin{center}
\textbf{Bringing this all together...}
\end{center}
\begin{itemize}
\item What`s important is the following code does not care if Payment is
\item ... an Interface
\item ... or an Abstract Class
\bigskip
\item ... because we refer to it by its supertype 
\item ... we know the method(s) we can call on this supertype (specified in the interface or abstract class)
\item ... and therefore implemented (or inherited) in all of its subtypes
\end{itemize}
\end{frame}

\begin{frame}[fragile]
\begin{block}{}
\begin{lstlisting}
public class NozamaUserDetails{
    
    private String name;
    private Payment paymentMethod;
    
    public NozamaUserDetails(String name, Payment paymentMethod){
    	this.name = name;
	this.paymentMethod = paymentMethod;
    }
    
    // other methods...
}
\end{lstlisting}
\end{block}
\end{frame}

\begin{frame}[fragile]
\begin{block}{}
\begin{lstlisting}
public class NozamaRegisterUser{
    
   public NozamaUserDetails registerUser(String name){
        Payment p = getPaymentMethod();
        return new NozamaUserDetails(name, paymentMethod);
    }
    
    private Payment getPaymentMethod(){
    	return new PayPal();
    }
}
\end{lstlisting}
\end{block}
\end{frame}

\begin{frame}[fragile]
\begin{block}{}
\begin{lstlisting}
public class ProcessPayment{
       
    public void purchase(NozamaUserDetails user, Item item){
    	Payment p = user.getPaymentMethod();
	p.makePayment(item.getPrice());
    }
    
    // other methods...
}
\end{lstlisting}
\end{block}
\end{frame}

\begin{frame}[fragile]
\begin{block}{}
\begin{lstlisting}
public class Ping implements Payment{
    public void makePayment(double debit) {
        // logic for Ping payment
        // e.g. mobile phone number
        // ... and authorising with bank ?? 
    }
}

public class Ping extends Payment{
    public void makePayment(double debit) {
        // logic for Ping payment
        // e.g. mobile phone number
        super.authoriseWithBank(authToken);
    }
}
\end{lstlisting}
\end{block}
\end{frame}

% *** Choosing An Interface or Abstract Class
\begin{frame}
\begin{center}
\textbf{Interfaces vs Abstract Classes}
\end{center}
\begin{itemize}
\item It can be difficult to identify 
\begin{itemize}
\item when to use an abstract class
\item when to use an interface
\end{itemize}
\item As a simple rule of thumb, when faced with a choice between abstract classes and interfaces
\begin{itemize}
\item use an abstract class when
\item you want to implement some but not all of a class's methods
\item and you are willing to accept the restrictions imposed upon classes
\item e.g. single inheritance
\item otherwise use an interface
\end{itemize}
\end{itemize}
\end{frame}

% *** Finally ***
\begin{frame}
\begin{center}
\textbf{Choosing to extend a \\ Class or Abstract Class \\ or Implement an Interface}
\end{center}
\end{frame}

\begin{frame}
\begin{center}
\textbf{Interfaces Over Abstract Classes or Classes}
\end{center}
\begin{itemize}
\item Want to include methods in superclasses but only allow them to be used by subclasses
\item Know what we want but not how to do it
\end{itemize}
\end{frame}

\begin{frame}
\begin{center}
\textbf{Abstract Classes Over Interfaces}
\end{center}
\begin{itemize}
\item We know some of want we want to do (i.e. can provide common functionality) ...
\item ... but not all
\end{itemize}
\end{frame}

\begin{frame}
\begin{center}
\textbf{Classes Over Abstract Classes}
\end{center}
\begin{itemize}
\item When would we choose to override methods rather than declaring them abstract?
\bigskip
\item ... lets think about the DOME example
\item ... real world, physical things on a Book Shelf
\item ... and a Database to keep track of my stuff
\end{itemize}
\end{frame}

%\begin{frame}
%\begin{center}
%\textbf{Final Examples}
%\end{center}
%\end{frame}

% *** HERE ***
\begin{frame}
\begin{center}
\textbf{Summary}
\end{center}
\begin{itemize}
\item Inheritance can provide shared implementation...
\item ... both as concrete and abstract classes
\item Inheritance provides shared type information
\item ... interfaces
\end{itemize}
\end{frame}

\begin{frame}
\begin{center}
\textbf{Summary}
\end{center}
\begin{itemize}
\item Abstract classes function as incomplete superclasses
\item Interfaces provide specification without implementation
\bigskip
\item Both Interfaces and Abstract Classes support polymorphism
\end{itemize}
\end{frame}

\end{document}